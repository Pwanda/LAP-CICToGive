\documentclass[a4paper,12pt]{article}
\usepackage[utf8]{inputenc}
\usepackage[ngerman]{babel}
\usepackage{geometry}
\usepackage{fancyhdr}
\usepackage{graphicx}
\usepackage{booktabs}
\usepackage{enumitem}
\usepackage{xcolor}
\usepackage{hyperref}
\usepackage{longtable}
\usepackage{array}
\usepackage{tabularx}
\usepackage{tikz}
\usepackage{pgfgantt}
\usepackage{rotating}
\usepackage{ltxtable}

\geometry{a4paper, left=3cm, right=2cm, top=2.5cm, bottom=2.5cm}

\pagestyle{fancy}
\fancyhf{}
\fancyhead[L]{LAP - Verschenke-Plattform}
\fancyhead[R]{Pflichtenheft}
\fancyfoot[C]{\thepage}

\hypersetup{
    colorlinks=true,
    linkcolor=blue,
    filecolor=magenta,
    urlcolor=cyan,
    pdftitle={Pflichtenheft - LAP Verschenke-Plattform},
    pdfauthor={Prüfungskandidat},
    pdfsubject={Lehrabschlussprüfung Applikationsentwicklung},
    pdfkeywords={Pflichtenheft, Implementierung, Zeitschätzung, Full-Stack}
}

\title{
    \huge\textbf{Pflichtenheft}\\
    \Large LAP - Verschenke-Plattform\\
    \large Technische Spezifikation und Implementierungsplan
}
\author{Lehrabschlussprüfung Applikationsentwicklung}
\date{\today}

\begin{document}

\maketitle

\newpage

\tableofcontents

\newpage

\section{Einleitung}

\subsection{Zweck des Dokuments}
Dieses Pflichtenheft definiert die technische Implementierung der LAP-Verschenke-Plattform basierend auf den im Lastenheft definierten Anforderungen. Es enthält detaillierte Spezifikationen, Architekturentscheidungen und eine realistische Zeitschätzung für die Entwicklung.

\subsection{Projektübersicht}
Die LAP-Plattform ist eine moderne Full-Stack-Webanwendung, die es Benutzern ermöglicht, Gegenstände zu verschenken und zu finden. Das System besteht aus einem React-Frontend, einem Spring Boot-Backend und einer PostgreSQL-Datenbank, ergänzt durch Cloud-Speicher für Dateien.

\subsection{Technische Zielsetzung}
\begin{itemize}
    \item Entwicklung einer skalierbaren, wartbaren Full-Stack-Anwendung
    \item Implementierung moderner Entwicklungsstandards und Best Practices
    \item Gewährleistung hoher Sicherheitsstandards
    \item Optimierung für Performance und Benutzerfreundlichkeit
    \item Containerisierte Entwicklungs- und Deployment-Umgebung
\end{itemize}

\section{Systemarchitektur}

\subsection{Architektur-Überblick}
Das System folgt einer dreischichtigen Architektur:

\begin{itemize}
    \item \textbf{Präsentationsschicht:} React-Frontend mit TypeScript
    \item \textbf{Anwendungsschicht:} Spring Boot-Backend mit REST-API
    \item \textbf{Datenschicht:} PostgreSQL-Datenbank mit JPA/Hibernate
    \item \textbf{Speicherschicht:} Backblaze B2 Cloud Storage für Dateien
\end{itemize}

\subsection{Technische Komponenten}

\subsubsection{Frontend-Architektur}
\begin{longtable}{|p{0.25\textwidth}|p{0.3\textwidth}|p{0.35\textwidth}|}
\hline
\textbf{Komponente} & \textbf{Technologie} & \textbf{Zweck} \\
\hline
UI-Framework & React 19 + TypeScript & Moderne, typsichere Benutzeroberfläche \\
\hline
Build-Tool & Vite & Schnelle Entwicklung und Builds \\
\hline
Routing & React Router v7 & Client-seitiges Routing \\
\hline
State Management & React Query & Server-State-Management und Caching \\
\hline
Styling & TailwindCSS + DaisyUI & Utility-first CSS mit Komponenten \\
\hline
Formulare & React Hook Form + Zod & Effiziente Formularvalidierung \\
\hline
HTTP-Client & Axios & API-Kommunikation \\
\hline
\end{longtable}

\subsubsection{Backend-Architektur}
\begin{longtable}{|p{0.25\textwidth}|p{0.3\textwidth}|p{0.35\textwidth}|}
\hline
\textbf{Komponente} & \textbf{Technologie} & \textbf{Zweck} \\
\hline
Framework & Spring Boot 3.2 & Enterprise-Java-Framework \\
\hline
Sicherheit & Spring Security & Authentifizierung und Autorisierung \\
\hline
Datenzugriff & Spring Data JPA & ORM und Datenbankzugriff \\
\hline
Authentifizierung & JWT & Token-basierte Authentifizierung \\
\hline
Validierung & Bean Validation & Server-seitige Validierung \\
\hline
Dokumentation & Spring Boot Actuator & Monitoring und Health Checks \\
\hline
\end{longtable}

\subsubsection{Datenbank-Design}
\begin{longtable}{|p{0.2\textwidth}|p{0.35\textwidth}|p{0.35\textwidth}|}
\hline
\textbf{Tabelle} & \textbf{Zweck} & \textbf{Hauptfelder} \\
\hline
users & Benutzerverwaltung & id, username, email, password\_hash, avatar\_url, created\_at, updated\_at \\
\hline
items & Artikelverwaltung & id, title, description, category, location, condition, is\_reserved, user\_id, created\_at \\
\hline
item\_images & Artikel-Bilder & item\_id, image\_url \\
\hline
comments & Kommentarsystem & id, content, item\_id, user\_id, created\_at \\
\hline
\end{longtable}

\section{Detaillierte Funktionsspezifikation}

\subsection{Authentifizierung und Autorisierung}

\subsubsection{Technische Umsetzung}
\begin{itemize}
    \item \textbf{Passwort-Verschlüsselung:} BCrypt mit Strength 12
    \item \textbf{JWT-Token:} HMAC-SHA256 mit 24h Ablaufzeit
    \item \textbf{Token-Speicherung:} Secure HttpOnly Cookies (Optional: localStorage)
    \item \textbf{Refresh-Token:} Implementierung für automatische Token-Erneuerung
\end{itemize}

\subsubsection{API-Endpunkte}
\begin{longtable}{|p{0.15\textwidth}|p{0.25\textwidth}|p{0.25\textwidth}|p{0.25\textwidth}|}
\hline
\textbf{Methode} & \textbf{Endpunkt} & \textbf{Zweck} & \textbf{Authentifizierung} \\
\hline
POST & /api/auth/register & Benutzerregistrierung & Nein \\
\hline
POST & /api/auth/login & Benutzeranmeldung & Nein \\
\hline
GET & /api/auth/me & Benutzerinfo abrufen & Ja \\
\hline
POST & /api/auth/logout & Benutzerabmeldung & Ja \\
\hline
PUT & /api/auth/profile & Profil aktualisieren & Ja \\
\hline
\end{longtable}

\subsection{Artikelverwaltung}

\subsubsection{Technische Umsetzung}
\begin{itemize}
    \item \textbf{Validierung:} Bean Validation mit Custom Constraints
    \item \textbf{Bildupload:} Multipart-Upload mit Größenbegrenzung (10MB)
    \item \textbf{Bildverarbeitung:} Automatische Größenanpassung
    \item \textbf{Suchfunktion:} JPA Criteria API für dynamische Queries
\end{itemize}

\subsubsection{API-Endpunkte}
\begin{longtable}{|p{0.15\textwidth}|p{0.25\textwidth}|p{0.25\textwidth}|p{0.25\textwidth}|}
\hline
\textbf{Methode} & \textbf{Endpunkt} & \textbf{Zweck} & \textbf{Authentifizierung} \\
\hline
GET & /api/items & Artikel auflisten & Nein \\
\hline
GET & /api/items/\{id\} & Artikel abrufen & Nein \\
\hline
POST & /api/items & Artikel erstellen & Ja \\
\hline
PUT & /api/items/\{id\} & Artikel aktualisieren & Ja (Besitzer) \\
\hline
DELETE & /api/items/\{id\} & Artikel löschen & Ja (Besitzer) \\
\hline
GET & /api/items/search & Artikel suchen & Nein \\
\hline
PUT & /api/items/\{id\}/reserve & Artikel reservieren & Ja \\
\hline
GET & /api/items/my & Eigene Artikel & Ja \\
\hline
\end{longtable}

\subsection{Dateiverwaltung}

\subsubsection{Technische Umsetzung}
\begin{itemize}
    \item \textbf{Cloud-Speicher:} Backblaze B2 Integration
    \item \textbf{Datei-Upload:} Chunked Upload für große Dateien
    \item \textbf{Bildoptimierung:} Automatische Komprimierung
    \item \textbf{URL-Generierung:} Sichere, zeitlich begrenzte URLs
\end{itemize}

\subsubsection{API-Endpunkte}
\begin{longtable}{|p{0.15\textwidth}|p{0.25\textwidth}|p{0.25\textwidth}|p{0.25\textwidth}|}
\hline
\textbf{Methode} & \textbf{Endpunkt} & \textbf{Zweck} & \textbf{Authentifizierung} \\
\hline
POST & /api/files/upload & Datei hochladen & Ja \\
\hline
GET & /api/files/\{filename\} & Datei abrufen & Nein \\
\hline
DELETE & /api/files/\{filename\} & Datei löschen & Ja \\
\hline
GET & /api/files/url/\{filename\} & Download-URL generieren & Nein \\
\hline
\end{longtable}

\section{Frontend-Implementierung}

\subsection{Komponentenstruktur}
\begin{longtable}{|p{0.3\textwidth}|p{0.35\textwidth}|p{0.25\textwidth}|}
\hline
\textbf{Komponente} & \textbf{Zweck} & \textbf{Typ} \\
\hline
App.tsx & Hauptapplikation mit Routing & Container \\
\hline
Navbar.tsx & Navigation und Benutzermenü & Layout \\
\hline
HomePage.tsx & Artikelübersicht mit Suche & Page \\
\hline
ItemCard.tsx & Artikel-Kachel in der Übersicht & Component \\
\hline
ItemModal.tsx & Artikel-Detailansicht & Modal \\
\hline
CreateItemModal.tsx & Artikel-Erstellungsformular & Modal \\
\hline
LoginPage.tsx & Anmeldungsformular & Page \\
\hline
RegisterPage.tsx & Registrierungsformular & Page \\
\hline
ProfilePage.tsx & Benutzerprofil & Page \\
\hline
ProtectedRoute.tsx & Routenschutz & HOC \\
\hline
\end{longtable}

\subsection{State Management}
\begin{itemize}
    \item \textbf{Server State:} React Query für API-Daten
    \item \textbf{Authentication State:} React Context für Benutzerstatus
    \item \textbf{Form State:} React Hook Form für Formulare
    \item \textbf{UI State:} useState für lokale Komponenten-States
\end{itemize}

\subsection{Responsive Design}
\begin{longtable}{|p{0.2\textwidth}|p{0.25\textwidth}|p{0.45\textwidth}|}
\hline
\textbf{Breakpoint} & \textbf{Größe} & \textbf{Layout-Anpassungen} \\
\hline
Mobile & < 768px & Single-column Layout, Burger-Menü, Touch-optimiert \\
\hline
Tablet & 768px - 1024px & 2-column Grid, kompakte Navigation \\
\hline
Desktop & > 1024px & 3-4 column Grid, volle Navigation \\
\hline
Large Desktop & > 1280px & 4+ column Grid, maximale Raumausnutzung \\
\hline
\end{longtable}

\section{Backend-Implementierung}

\subsection{Schichtarchitektur}
\begin{longtable}{|p{0.2\textwidth}|p{0.35\textwidth}|p{0.35\textwidth}|}
\hline
\textbf{Schicht} & \textbf{Zweck} & \textbf{Komponenten} \\
\hline
Controller & REST-API Endpunkte & AuthController, ItemController, FileController \\
\hline
Service & Business Logic & AuthService, ItemService, FileStorageService \\
\hline
Repository & Datenzugriff & UserRepository, ItemRepository, CommentRepository \\
\hline
Entity & Datenmodell & User, Item, Comment \\
\hline
DTO & Datentransfer & AuthDTO, ItemDTO, CommentDTO \\
\hline
Configuration & Konfiguration & SecurityConfig, JwtUtil, WebConfig \\
\hline
\end{longtable}

\subsection{Sicherheitskonzept}
\begin{itemize}
    \item \textbf{CORS-Konfiguration:} Whitelist für erlaubte Origins
    \item \textbf{JWT-Filter:} Automatische Token-Validierung
    \item \textbf{Method-Level Security:} @PreAuthorize für granulare Kontrolle
    \item \textbf{Input Validation:} Umfassende Validierung aller Eingaben
    \item \textbf{Error Handling:} Sichere Fehlerbehandlung ohne Informationslecks
\end{itemize}

\subsection{Datenbankoptimierung}
\begin{itemize}
    \item \textbf{Indizierung:} Strategische Indizes für Suchfelder
    \item \textbf{Lazy Loading:} Optimierte JPA-Beziehungen
    \item \textbf{Query Optimization:} Effiziente JPA-Queries
    \item \textbf{Connection Pooling:} HikariCP für Performance
\end{itemize}

\section{Entwicklungsumgebung}

\subsection{Container-Setup}
\begin{longtable}{|p{0.2\textwidth}|p{0.25\textwidth}|p{0.45\textwidth}|}
\hline
\textbf{Container} & \textbf{Purpose} & \textbf{Konfiguration} \\
\hline
lap-frontend & React Development & Node.js 18, Vite Dev Server, Hot Reload \\
\hline
lap-backend & Spring Boot API & Java 17, Maven, Spring Boot DevTools \\
\hline
lap-postgres & Database & PostgreSQL 14, persistent volume \\
\hline
lap-pgadmin & DB Admin & PgAdmin 4, web interface \\
\hline
\end{longtable}

\subsection{Development Workflow}
\begin{itemize}
    \item \textbf{Hot Reload:} Frontend und Backend automatische Neulades
    \item \textbf{Volume Mounting:} Live-Codeänderungen
    \item \textbf{Database Seeding:} Automatische Testdaten
    \item \textbf{Port Forwarding:} Lokale Entwicklungsports
\end{itemize}

\section{Testkonzept}

\subsection{Frontend-Tests}
\begin{longtable}{|p{0.25\textwidth}|p{0.35\textwidth}|p{0.3\textwidth}|}
\hline
\textbf{Test-Typ} & \textbf{Framework} & \textbf{Abdeckung} \\
\hline
Unit Tests & Jest + React Testing Library & Komponenten-Logic \\
\hline
Integration Tests & Cypress & User Workflows \\
\hline
Visual Tests & Storybook & Component Gallery \\
\hline
\end{longtable}

\subsection{Backend-Tests}
\begin{longtable}{|p{0.25\textwidth}|p{0.35\textwidth}|p{0.3\textwidth}|}
\hline
\textbf{Test-Typ} & \textbf{Framework} & \textbf{Abdeckung} \\
\hline
Unit Tests & JUnit 5 + Mockito & Service Layer \\
\hline
Integration Tests & Spring Boot Test & API Endpoints \\
\hline
Repository Tests & @DataJpaTest & Database Operations \\
\hline
Security Tests & Spring Security Test & Authentication \\
\hline
\end{longtable}

\section{Deployment und DevOps}

\subsection{Containerisierung}
\begin{itemize}
    \item \textbf{Multi-Stage Builds:} Optimierte Production Images
    \item \textbf{Health Checks:} Container-Überwachung
    \item \textbf{Volume Management:} Persistent Storage
    \item \textbf{Network Configuration:} Sichere Container-Kommunikation
\end{itemize}

\subsection{CI/CD Pipeline}
\begin{itemize}
    \item \textbf{Build Stage:} Automatische Builds bei Code-Änderungen
    \item \textbf{Test Stage:} Ausführung aller Tests
    \item \textbf{Security Scan:} Dependency-Vulnerability-Checks
    \item \textbf{Deploy Stage:} Automatisches Deployment
\end{itemize}

\section{Zeitschätzung und Projektplanung}

\subsection{Entwicklungsphasen}
\begin{longtable}{|p{0.05\textwidth}|p{0.3\textwidth}|p{0.15\textwidth}|p{0.4\textwidth}|}
\hline
\textbf{Nr.} & \textbf{Phase} & \textbf{Dauer} & \textbf{Aktivitäten} \\
\hline
1 & Projektsetup & 8 Stunden & Repository Setup, Entwicklungsumgebung, Toolchain \\
\hline
2 & Backend-Grundstruktur & 16 Stunden & Spring Boot Setup, Datenbankmodell, Grundkonfiguration \\
\hline
3 & Authentifizierung & 12 Stunden & JWT-Implementierung, User-Management, Security \\
\hline
4 & Frontend-Grundstruktur & 12 Stunden & React Setup, Routing, Basic Components \\
\hline
5 & Artikelverwaltung Backend & 20 Stunden & CRUD-Operationen, Validierung, Tests \\
\hline
6 & Artikelverwaltung Frontend & 24 Stunden & UI-Komponenten, Formulare, State Management \\
\hline
7 & Dateiverwaltung & 16 Stunden & B2-Integration, Upload-Funktionalität \\
\hline
8 & Suchfunktionalität & 12 Stunden & Backend-Suche, Frontend-Filter, Paginierung \\
\hline
9 & Kommentarsystem & 10 Stunden & Backend-APIs, Frontend-Komponenten \\
\hline
10 & UI/UX-Verbesserungen & 16 Stunden & Responsive Design, Styling, Accessibility \\
\hline
11 & Testing & 20 Stunden & Unit-Tests, Integration-Tests, E2E-Tests \\
\hline
12 & Deployment & 8 Stunden & Containerisierung, CI/CD, Production-Setup \\
\hline
13 & Dokumentation & 12 Stunden & API-Docs, Benutzerhandbuch, Entwickler-Docs \\
\hline
14 & Bugfixing \& Optimierung & 16 Stunden & Performance-Optimierung, Bugfixes \\
\hline
& \textbf{Gesamt} & \textbf{182 Stunden} & \textbf{≈ 23 Arbeitstage} \\
\hline
\end{longtable}

\subsection{Detaillierte Zeitschätzung nach Bereichen}

\subsubsection{Backend-Entwicklung (84 Stunden)}
\begin{longtable}{|p{0.4\textwidth}|p{0.15\textwidth}|p{0.35\textwidth}|}
\hline
\textbf{Aufgabe} & \textbf{Stunden} & \textbf{Details} \\
\hline
Spring Boot Setup \& Konfiguration & 8 & Projekt-Setup, Dependencies, Properties \\
\hline
Datenbankmodell \& JPA-Entities & 10 & User, Item, Comment Entities, Relationships \\
\hline
Security \& JWT-Implementation & 12 & Spring Security, JWT-Filter, Authentication \\
\hline
User-Management APIs & 8 & Registration, Login, Profile-Management \\
\hline
Item-Management APIs & 16 & CRUD-Operationen, Validierung, Business Logic \\
\hline
File-Storage Integration & 12 & B2-Integration, Upload/Download APIs \\
\hline
Search \& Filter APIs & 8 & JPA Criteria, Dynamic Queries \\
\hline
Comment-System APIs & 6 & CRUD für Kommentare \\
\hline
Testing \& Debugging & 4 & Unit-Tests, Integration-Tests \\
\hline
\end{longtable}

\subsubsection{Frontend-Entwicklung (72 Stunden)}
\begin{longtable}{|p{0.4\textwidth}|p{0.15\textwidth}|p{0.35\textwidth}|}
\hline
\textbf{Aufgabe} & \textbf{Stunden} & \textbf{Details} \\
\hline
React Setup \& Toolchain & 6 & Vite, TypeScript, ESLint, Prettier \\
\hline
Routing \& Navigation & 8 & React Router, Protected Routes, Navigation \\
\hline
Authentication Components & 12 & Login, Register, Profile Components \\
\hline
Item-Management UI & 20 & ItemCard, ItemModal, CreateItemModal, EditItemModal \\
\hline
Search \& Filter UI & 8 & Searchbar, Category Filter, Pagination \\
\hline
File-Upload Components & 6 & Image Upload, Preview, Drag \& Drop \\
\hline
Responsive Design & 8 & Mobile Optimization, Tablet Layout \\
\hline
State Management & 4 & React Query Setup, Context API \\
\hline
\end{longtable}

\subsubsection{DevOps \& Deployment (26 Stunden)}
\begin{longtable}{|p{0.4\textwidth}|p{0.15\textwidth}|p{0.35\textwidth}|}
\hline
\textbf{Aufgabe} & \textbf{Stunden} & \textbf{Details} \\
\hline
Containerisierung & 8 & Docker/Podman Setup, Multi-Container Environment \\
\hline
Development Environment & 6 & docker-compose, Hot Reload, Volume Mounting \\
\hline
CI/CD Pipeline & 4 & GitHub Actions, Automated Testing \\
\hline
Production Deployment & 4 & Production Configuration, Environment Variables \\
\hline
Monitoring \& Logging & 4 & Health Checks, Log Aggregation \\
\hline
\end{longtable}

\subsection{Projektphasen-Gantt-Chart}

\begin{sidewaystable}
\centering
\begin{ganttchart}[
    y unit title=0.5cm,
    y unit chart=0.5cm,
    vgrid,
    hgrid,
    title label font=\bfseries\footnotesize,
    bar label font=\footnotesize,
    milestone label font=\footnotesize,
    group label font=\footnotesize
    ]{1}{23}
\gantttitle{Entwicklungszeitraum (Arbeitstage)}{23} \\
\gantttitlelist{1,...,23}{1} \\
\ganttgroup{Setup \& Grundlagen}{1}{4} \\
\ganttbar{Projektsetup}{1}{1} \\
\ganttbar{Backend-Grundstruktur}{2}{4} \\
\ganttgroup{Authentifizierung}{5}{7} \\
\ganttbar{JWT \& Security}{5}{6} \\
\ganttbar{Frontend Auth}{7}{7} \\
\ganttgroup{Kernfunktionalität}{8}{16} \\
\ganttbar{Backend APIs}{8}{11} \\
\ganttbar{Frontend Components}{9}{13} \\
\ganttbar{File Management}{14}{16} \\
\ganttgroup{Erweiterte Features}{17}{19} \\
\ganttbar{Suche \& Filter}{17}{18} \\
\ganttbar{Kommentare}{19}{19} \\
\ganttgroup{Qualitätssicherung}{20}{23} \\
\ganttbar{Testing}{20}{21} \\
\ganttbar{UI/UX Polish}{22}{22} \\
\ganttbar{Deployment}{23}{23} \\
\ganttmilestone{MVP Release}{16} \\
\ganttmilestone{Final Release}{23}
\end{ganttchart}
\end{sidewaystable}

\subsection{Risiko-Puffer}
\begin{itemize}
    \item \textbf{Technische Risiken:} +20\% (36 Stunden) für unvorhergesehene Probleme
    \item \textbf{Integration-Challenges:} +10\% (18 Stunden) für Integrationsprobleme
    \item \textbf{Learning Curve:} +15\% (27 Stunden) für neue Technologien
    \item \textbf{Gesamt-Puffer:} +45\% (81 Stunden)
\end{itemize}

\textbf{Realistische Gesamtschätzung:} 182 + 81 = \textbf{263 Stunden (≈ 33 Arbeitstage)}

\section{Qualitätskriterien}

\subsection{Code-Qualität}
\begin{longtable}{|p{0.3\textwidth}|p{0.3\textwidth}|p{0.3\textwidth}|}
\hline
\textbf{Kriterium} & \textbf{Zielwert} & \textbf{Messverfahren} \\
\hline
Test-Abdeckung & > 80\% & SonarQube, Coverage Reports \\
\hline
Code-Komplexität & < 10 (Cyclomatic) & ESLint, SonarQube \\
\hline
Duplicate Code & < 5\% & SonarQube \\
\hline
TypeScript-Typisierung & 100\% & TypeScript Compiler \\
\hline
Code-Style & 100\% Konformität & ESLint, Prettier \\
\hline
\end{longtable}

\subsection{Performance-Kriterien}
\begin{longtable}{|p{0.3\textwidth}|p{0.3\textwidth}|p{0.3\textwidth}|}
\hline
\textbf{Kriterium} & \textbf{Zielwert} & \textbf{Messverfahren} \\
\hline
Seitenladezeit & < 2 Sekunden & Lighthouse, WebPageTest \\
\hline
API-Antwortzeit & < 500ms & Application Metrics \\
\hline
Bildladezeit & < 3 Sekunden & Browser DevTools \\
\hline
First Contentful Paint & < 1.5 Sekunden & Lighthouse \\
\hline
Bundle-Größe & < 1MB gzipped & Webpack Bundle Analyzer \\
\hline
\end{longtable}

\subsection{Sicherheitskriterien}
\begin{longtable}{|p{0.3\textwidth}|p{0.3\textwidth}|p{0.3\textwidth}|}
\hline
\textbf{Kriterium} & \textbf{Zielwert} & \textbf{Messverfahren} \\
\hline
Vulnerability Scan & 0 High-Risk Issues & OWASP ZAP, Snyk \\
\hline
Authentication Security & JWT Best Practices & Security Review \\
\hline
Input Validation & 100\% Coverage & Code Review \\
\hline
HTTPS Enforcement & 100\% & SSL Labs Test \\
\hline
Dependency Security & 0 Known Vulnerabilities & npm audit, OWASP \\
\hline
\end{longtable}

\section{Wartung und Erweiterbarkeit}

\subsection{Wartungskonzept}
\begin{itemize}
    \item \textbf{Dependency Updates:} Monatliche Updates für Security-Patches
    \item \textbf{Database Maintenance:} Regelmäßige Backups und Performance-Optimierung
    \item \textbf{Monitoring:} Kontinuierliche Überwachung von Performance und Fehlern
    \item \textbf{Log Analysis:} Wöchentliche Analyse der Anwendungslogs
    \item \textbf{Security Updates:} Sofortige Patches für kritische Sicherheitslücken
\end{itemize}

\subsection{Erweiterbarkeit}
\begin{itemize}
    \item \textbf{Modulare Architektur:} Neue Features können unabhängig entwickelt werden
    \item \textbf{API-First Design:} Einfache Integration neuer Clients
    \item \textbf{Microservice-Ready:} Architektur erlaubt spätere Aufteilung
    \item \textbf{Database Migration:} Liquibase/Flyway für Schema-Updates
    \item \textbf{Feature Flags:} Graduelle Rollouts neuer Funktionen
\end{itemize}

\section{Technische Risiken und Mitigationen}

\subsection{Identifizierte Risiken}
\begin{longtable}{|p{0.25\textwidth}|p{0.2\textwidth}|p{0.25\textwidth}|p{0.2\textwidth}|}
\hline
\textbf{Risiko} & \textbf{Wahrscheinlichkeit} & \textbf{Auswirkung} & \textbf{Mitigation} \\
\hline
Performance-Probleme bei großen Datenmengen & Mittel & Hoch & Indexierung, Pagination, Caching \\
\hline
Sicherheitslücken in Dependencies & Hoch & Hoch & Regelmäßige Scans, Updates \\
\hline
B2-Service-Ausfälle & Niedrig & Mittel & Retry-Logic, Fallback-Strategie \\
\hline
Browser-Kompatibilität & Niedrig & Mittel & Polyfills, Progressive Enhancement \\
\hline
Datenbankperformance & Mittel & Hoch & Query-Optimierung, Connection Pooling \\
\hline
\end{longtable}

\subsection{Backup und Recovery}
\begin{itemize}
    \item \textbf{Datenbank-Backups:} Tägliche automatische Backups
    \item \textbf{File-Storage:} B2-native Redundanz und Versionierung
    \item \textbf{Code-Repository:} Git-basierte Versionskontrolle
    \item \textbf{Configuration:} Infrastructure as Code für Reproduzierbarkeit
    \item \textbf{Disaster Recovery:} Dokumentierte Wiederherstellungsverfahren
\end{itemize}

\section{Erfolgsmetriken}

\subsection{Technische Metriken}
\begin{longtable}{|p{0.3\textwidth}|p{0.25\textwidth}|p{0.35\textwidth}|}
\hline
\textbf{Metrik} & \textbf{Zielwert} & \textbf{Messverfahren} \\
\hline
Systemverfügbarkeit & > 99.5\% & Uptime Monitoring \\
\hline
Durchschnittliche Antwortzeit & < 500ms & Application Performance Monitoring \\
\hline
Fehlerrate & < 1\% & Error Tracking (Sentry) \\
\hline
Deployment-Häufigkeit & Täglich möglich & CI/CD Pipeline Metriken \\
\hline
Mean Time to Recovery & < 30 Minuten & Incident Response Tracking \\
\hline
\end{longtable}

\subsection{Qualitätsmetriken}
\begin{longtable}{|p{0.3\textwidth}|p{0.25\textwidth}|p{0.35\textwidth}|}
\hline
\textbf{Metrik} & \textbf{Zielwert} & \textbf{Messverfahren} \\
\hline
Code Coverage & > 80\% & JaCoCo, Jest Coverage \\
\hline
Sonar Quality Gate & Passed & SonarQube Analysis \\
\hline
Security Vulnerabilities & 0 Critical & OWASP ZAP, Snyk \\
\hline
Performance Budget & < 2MB Bundle & Webpack Bundle Analyzer \\
\hline
Accessibility Score & > 90\% & Lighthouse Accessibility Audit \\
\hline
\end{longtable}

\section{Implementierungsrichtlinien}

\subsection{Coding Standards}
\begin{itemize}
    \item \textbf{Java:} Google Java Style Guide, Checkstyle
    \item \textbf{TypeScript:} Airbnb Style Guide, ESLint
    \item \textbf{Git:} Conventional Commits, Branch Protection
    \item \textbf{Documentation:} JavaDoc, TSDoc, README-Standards
    \item \textbf{Code Review:} Minimum 2 Reviewer, Automated Checks
\end{itemize}

\subsection{Development Best Practices}
\begin{itemize}
    \item \textbf{TDD:} Test-Driven Development für kritische Funktionen
    \item \textbf{CI/CD:} Continuous Integration und Deployment
    \item \textbf{Feature Branches:} GitFlow-Workflow
    \item \textbf{Code Reviews:} Peer Reviews für alle Änderungen
    \item \textbf{Documentation:} Living Documentation, ADRs
\end{itemize}

\section{Fazit}

\subsection{Projektzusammenfassung}
Die LAP-Verschenke-Plattform stellt eine technisch anspruchsvolle Full-Stack-Anwendung dar, die moderne Entwicklungsstandards und Best Practices implementiert. Die geschätzte Entwicklungszeit von 263 Stunden (33 Arbeitstage) berücksichtigt alle Aspekte der Softwareentwicklung von der Planung bis zum Deployment.

\subsection{Technische Highlights}
\begin{itemize}
    \item Moderne React-Architektur mit TypeScript
    \item Robuste Spring Boot-Backend-Implementierung
    \item Sichere JWT-basierte Authentifizierung
    \item Skalierbare Cloud-Speicher-Integration
    \item Containerisierte Entwicklungsumgebung
    \item Comprehensive Testing-Strategie
\end{itemize}

\subsection{Lernziele}
Dieses Projekt demonstriert:
\begin{itemize}
    \item Vollständige Full-Stack-Entwicklung
    \item Integration moderner Technologien
    \item Sicherheitsbewusstsein in der Entwicklung
    \item DevOps und Deployment-Praktiken
    \item Qualitätssicherung und Testing
    \item Projektmanagement und Zeitschätzung
\end{itemize}

\section{Anhang}

\subsection{Technologie-Versionen}
\begin{longtable}{|p{0.3\textwidth}|p{0.2\textwidth}|p{0.4\textwidth}|}
\hline
\textbf{Technologie} & \textbf{Version} & \textbf{Begründung} \\
\hline
React & 19.x & Neueste stabile Version, moderne Features \\
\hline
TypeScript & 5.x & Beste Typisierung, moderne ECMAScript-Features \\
\hline
Spring Boot & 3.2.x & LTS-Version, Java 17 Unterstützung \\
\hline
PostgreSQL & 14.x & Stabile Version, JSON-Unterstützung \\
\hline
Node.js & 18.x LTS & Langzeit-Support, npm-Kompatibilität \\
\hline
Java & 17 LTS & Langzeit-Support, moderne Features \\
\hline
\end{longtable}

\subsection{Externe Abhängigkeiten}
\begin{itemize}
    \item \textbf{Backblaze B2:} Cloud-Speicher für Dateien
    \item \textbf{JWT Libraries:} Token-basierte Authentifizierung
    \item \textbf{BCrypt:} Passwort-Hashing
    \item \textbf{Hibernate:} ORM-Framework
    \item \textbf{React Query:} Server-State-Management
\end{itemize}

\subsection{Referenzen}
\begin{itemize}
    \item Spring Boot Best Practices: \url{https://spring.io/guides}
    \item React Performance Patterns: \url{https://reactjs.org/docs/optimizing-performance.html}
    \item JWT Security Guidelines: \url{https://tools.ietf.org/html/rfc7519}
    \item PostgreSQL Performance Tuning: \url{https://www.postgresql.org/docs/current/performance-tips.html}
    \item Container Security Best Practices: \url{https://docs.docker.com/develop/security-best-practices/}
\end{itemize}

\end{document}
