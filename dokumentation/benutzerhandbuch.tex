\documentclass[a4paper,12pt]{article}
\usepackage[utf8]{inputenc}
\usepackage[ngerman]{babel}
\usepackage{geometry}
\usepackage{fancyhdr}
\usepackage{graphicx}
\usepackage{booktabs}
\usepackage{enumitem}
\usepackage{xcolor}
\usepackage{hyperref}
\usepackage{longtable}
\usepackage{array}
\usepackage{tabularx}
\usepackage{listings}
\usepackage{framed}
\usepackage{tcolorbox}

\geometry{a4paper, left=3cm, right=2cm, top=2.5cm, bottom=2.5cm}

\pagestyle{fancy}
\fancyhf{}
\fancyhead[L]{LAP - Verschenke-Plattform}
\fancyhead[R]{Benutzerhandbuch}
\fancyfoot[C]{\thepage}

\hypersetup{
    colorlinks=true,
    linkcolor=blue,
    filecolor=magenta,
    urlcolor=cyan,
    pdftitle={Benutzerhandbuch - LAP Verschenke-Plattform},
    pdfauthor={Prüfungskandidat},
    pdfsubject={Lehrabschlussprüfung Applikationsentwicklung},
    pdfkeywords={Benutzerhandbuch, Installation, Bedienung, Full-Stack}
}

\lstset{
    basicstyle=\ttfamily\small,
    frame=single,
    backgroundcolor=\color{lightgray!20},
    breaklines=true,
    breakatwhitespace=true,
    tabsize=2,
    showstringspaces=false
}

\newtcolorbox{infobox}{
    colback=blue!5!white,
    colframe=blue!75!black,
    title=Information
}

\newtcolorbox{warningbox}{
    colback=orange!5!white,
    colframe=orange!75!black,
    title=Warnung
}

\newtcolorbox{tipbox}{
    colback=green!5!white,
    colframe=green!75!black,
    title=Tipp
}

\title{
    \huge\textbf{Benutzerhandbuch}\\
    \Large LAP - Verschenke-Plattform\\
    \large Installation, Konfiguration und Bedienung
}
\author{Lehrabschlussprüfung Applikationsentwicklung}
\date{\today}

\begin{document}

\maketitle

\newpage

\tableofcontents

\newpage

\section{Einleitung}

\subsection{Über die LAP-Verschenke-Plattform}
Die LAP-Verschenke-Plattform ist eine moderne Full-Stack-Webanwendung, die es Benutzern ermöglicht, nicht mehr benötigte Gegenstände zu verschenken, anstatt sie wegzuwerfen. Die Plattform fördert nachhaltiges Verhalten und schafft eine lokale Gemeinschaft zum Teilen von Ressourcen.

\subsection{Hauptfunktionen}
\begin{itemize}
    \item \textbf{Benutzerregistrierung und -anmeldung} mit sicherer Authentifizierung
    \item \textbf{Artikel erstellen} mit Bildern, Beschreibungen und Kategorien
    \item \textbf{Artikel suchen und filtern} nach verschiedenen Kriterien
    \item \textbf{Artikel reservieren} und Interesse bekunden
    \item \textbf{Kommentare} zu Artikeln hinzufügen
    \item \textbf{Persönliche Artikelverwaltung} für eigene Einträge
    \item \textbf{Profilverwaltung} mit Avatar-Upload
\end{itemize}

\subsection{Technische Übersicht}
\begin{itemize}
    \item \textbf{Frontend:} React 19 mit TypeScript, TailwindCSS und DaisyUI
    \item \textbf{Backend:} Spring Boot 3.2 mit Java 17
    \item \textbf{Datenbank:} PostgreSQL 14
    \item \textbf{Dateispeicher:} Backblaze B2 Cloud Storage
    \item \textbf{Containerisierung:} Podman/Docker für einfache Entwicklung
\end{itemize}

\section{Systemanforderungen}

\subsection{Mindestanforderungen}
\begin{longtable}{|p{0.3\textwidth}|p{0.6\textwidth}|}
\hline
\textbf{Komponente} & \textbf{Anforderung} \\
\hline
Betriebssystem & Windows 10/11, macOS 10.15+, Linux (Ubuntu 20.04+) \\
\hline
RAM & Mindestens 8 GB (16 GB empfohlen) \\
\hline
Festplatte & 5 GB freier Speicherplatz \\
\hline
Internetverbindung & Breitband für Cloud-Services \\
\hline
Webbrowser & Chrome 90+, Firefox 88+, Safari 14+, Edge 90+ \\
\hline
\end{longtable}

\subsection{Entwicklungsumgebung}
\begin{longtable}{|p{0.3\textwidth}|p{0.6\textwidth}|}
\hline
\textbf{Software} & \textbf{Version} \\
\hline
Podman oder Docker & Aktuelle Version \\
\hline
Git & 2.30+ \\
\hline
Node.js & 18.x LTS \\
\hline
Java & 17 LTS \\
\hline
Maven & 3.8+ \\
\hline
\end{longtable}

\section{Installation}

\subsection{Schnellstart für Entwickler}

\begin{infobox}
Die LAP-Plattform ist so konzipiert, dass neue Entwickler mit einem einzigen Befehl loslegen können!
\end{infobox}

\subsubsection{Voraussetzungen prüfen}
\begin{lstlisting}[language=bash]
# Podman installiert?
podman --version

# Git installiert?
git --version

# curl installiert?
curl --version
\end{lstlisting}

\subsubsection{Ein-Befehl-Setup}
\begin{lstlisting}[language=bash]
# Repository klonen
git clone <your-repo-url>
cd LAP

# Alles mit einem Befehl starten
./start-dev.sh
\end{lstlisting}

\begin{tipbox}
Das Skript erledigt automatisch:
\begin{itemize}
    \item Bau aller Container
    \item Einrichtung der PostgreSQL-Datenbank
    \item Erstellung eines Testbenutzers
    \item Start aller Services mit Hot Reload
\end{itemize}
\end{tipbox}

\subsection{Manuelle Installation}

\subsubsection{1. Repository klonen}
\begin{lstlisting}[language=bash]
git clone <your-repo-url>
cd LAP
\end{lstlisting}

\subsubsection{2. Container-Services starten}
\begin{lstlisting}[language=bash]
# Alle Services mit podman-compose starten
podman-compose up -d

# Oder mit Docker
docker-compose up -d
\end{lstlisting}

\subsubsection{3. Backend-Container bauen}
\begin{lstlisting}[language=bash]
# In Backend-Verzeichnis wechseln
cd backend

# Container bauen
podman build -t lap-backend .

# Container mit Datenbank verknüpfen
podman run -d --name lap-backend \
  --network lap-network \
  -p 8080:8080 \
  lap-backend
\end{lstlisting}

\subsubsection{4. Frontend-Container bauen}
\begin{lstlisting}[language=bash]
# In Frontend-Verzeichnis wechseln
cd Frontend

# Container bauen
podman build -t lap-frontend .

# Container starten
podman run -d --name lap-frontend \
  -p 5173:5173 \
  -v $(pwd)/src:/app/src \
  lap-frontend
\end{lstlisting}

\section{Konfiguration}

\subsection{Umgebungsvariablen}

\subsubsection{Backend-Konfiguration}
\begin{lstlisting}[language=properties]
# application.properties
spring.datasource.url=jdbc:postgresql://lap-postgres:5432/lapdb
spring.datasource.username=lapuser
spring.datasource.password=lappassword

# JWT-Konfiguration
jwt.secret=mySecretKey123456789012345678901234567890123456789012345678901234567890
jwt.expiration=86400000

# Backblaze B2 (optional)
b2.application.key.id=YOUR_B2_APPLICATION_KEY_ID
b2.application.key=YOUR_B2_APPLICATION_KEY
b2.bucket.name=YOUR_B2_BUCKET_NAME
\end{lstlisting}

\subsubsection{Frontend-Konfiguration}
\begin{lstlisting}[language=properties]
# .env
VITE_API_URL=http://localhost:8080
VITE_APP_NAME=LAP Verschenke-Plattform
\end{lstlisting}

\subsection{Datenbank-Setup}

\subsubsection{PostgreSQL-Verbindung prüfen}
\begin{lstlisting}[language=bash]
# Datenbankverbindung testen
podman exec lap-postgres pg_isready -U lapuser -d lapdb

# Datenbankshell öffnen
podman exec -it lap-postgres psql -U lapuser -d lapdb
\end{lstlisting}

\subsubsection{Testdaten erstellen}
\begin{lstlisting}[language=sql]
-- Testbenutzer erstellen (automatisch beim Start)
INSERT INTO users (username, email, password, created_at, updated_at)
VALUES ('testuser', 'test@example.com',
        '$2a$12$encrypted_password_hash',
        CURRENT_TIMESTAMP, CURRENT_TIMESTAMP);
\end{lstlisting}

\subsection{Cloud-Speicher (Backblaze B2)}

\subsubsection{B2-Account einrichten}
\begin{enumerate}
    \item Backblaze-Account erstellen: \url{https://www.backblaze.com/}
    \item B2 Cloud Storage aktivieren
    \item Neuen Bucket erstellen
    \item Application Key generieren
    \item Credentials in \texttt{application.properties} eintragen
\end{enumerate}

\begin{warningbox}
Credentials niemals in Git committen! Verwenden Sie Umgebungsvariablen für Production.
\end{warningbox}

\section{Bedienung}

\subsection{Erste Schritte}

\subsubsection{Zugriff auf die Anwendung}
Nach dem Start sind folgende URLs verfügbar:
\begin{itemize}
    \item \textbf{Frontend:} \url{http://localhost:5173}
    \item \textbf{Backend-API:} \url{http://localhost:8080}
    \item \textbf{PgAdmin:} \url{http://localhost:5050} (mit \texttt{--with-pgadmin})
\end{itemize}

\subsubsection{Testanmeldung}
\begin{itemize}
    \item \textbf{Benutzername:} testuser
    \item \textbf{Passwort:} password123
\end{itemize}

\subsection{Benutzerregistrierung}

\subsubsection{Neuen Account erstellen}
\begin{enumerate}
    \item Auf der Startseite auf \textbf{„Registrieren"} klicken
    \item Formular ausfüllen:
    \begin{itemize}
        \item Benutzername (3-50 Zeichen)
        \item E-Mail-Adresse
        \item Passwort (mindestens 6 Zeichen)
    \end{itemize}
    \item \textbf{„Registrieren"} klicken
    \item Automatische Weiterleitung zur Anmeldung
\end{enumerate}

\subsubsection{Anmeldung}
\begin{enumerate}
    \item Auf \textbf{„Anmelden"} klicken
    \item Benutzername und Passwort eingeben
    \item \textbf{„Anmelden"} klicken
    \item Weiterleitung zur Hauptseite
\end{enumerate}

\subsection{Hauptseite (Homepage)}

\subsubsection{Artikel-Übersicht}
Die Hauptseite zeigt alle verfügbaren Artikel in einer Kachel-Ansicht:
\begin{itemize}
    \item \textbf{Titel und Beschreibung} des Artikels
    \item \textbf{Kategorie} und \textbf{Standort}
    \item \textbf{Zustand} (Neu, Gut, Gebraucht)
    \item \textbf{Bilder} (falls vorhanden)
    \item \textbf{Reservierungsstatus}
\end{itemize}

\subsubsection{Suchfunktion}
\begin{enumerate}
    \item Suchbegriff in die Suchleiste eingeben
    \item Artikel werden automatisch gefiltert
    \item Suche erfolgt in Titel und Beschreibung
\end{enumerate}

\subsubsection{Kategoriefilter}
\begin{itemize}
    \item Klick auf Kategorie-Buttons filtert Artikel
    \item Verfügbare Kategorien: Elektronik, Möbel, Kleidung, Bücher, Spielzeug, Sonstiges
    \item \textbf{„Alle"} zeigt alle Artikel an
\end{itemize}

\subsubsection{Paginierung}
\begin{itemize}
    \item 12 Artikel pro Seite (3 Reihen × 4 Spalten)
    \item Navigation mit Vor/Zurück-Buttons
    \item Seitenzahlen für direkten Sprung
\end{itemize}

\subsection{Artikel erstellen}

\subsubsection{Neuen Artikel hinzufügen}
\begin{enumerate}
    \item \textbf{Plus-Button} unten rechts klicken
    \item Formular ausfüllen:
    \begin{itemize}
        \item \textbf{Titel:} Aussagekräftiger Name (max. 100 Zeichen)
        \item \textbf{Beschreibung:} Detaillierte Beschreibung (max. 500 Zeichen)
        \item \textbf{Kategorie:} Passende Kategorie auswählen
        \item \textbf{Standort:} Abholort angeben
        \item \textbf{Zustand:} Neu, Gut oder Gebraucht
        \item \textbf{Bilder:} Ein oder mehrere Bilder hochladen
    \end{itemize}
    \item \textbf{„Artikel erstellen"} klicken
\end{enumerate}

\begin{tipbox}
Gute Bilder erhöhen die Chance, dass Ihr Artikel gefunden wird!
\end{tipbox}

\subsubsection{Bildupload}
\begin{itemize}
    \item Unterstützte Formate: JPG, PNG, GIF
    \item Maximale Größe: 10 MB pro Bild
    \item Drag \& Drop oder Datei-Browser
    \item Mehrere Bilder gleichzeitig möglich
    \item Automatische Größenanpassung
\end{itemize}

\subsection{Artikel-Details}

\subsubsection{Artikel anzeigen}
\begin{enumerate}
    \item Artikel-Kachel anklicken
    \item Detail-Modal öffnet sich mit:
    \begin{itemize}
        \item Vollständigen Informationen
        \item Bildergalerie
        \item Kontakt zum Anbieter
        \item Kommentarbereich
    \end{itemize}
\end{enumerate}

\subsubsection{Artikel reservieren}
\begin{enumerate}
    \item In der Artikel-Detailansicht
    \item \textbf{„Reservieren"} klicken
    \item Artikel wird als „Reserviert" markiert
    \item Andere Nutzer sehen den Reservierungsstatus
\end{enumerate}

\subsubsection{Kommentare}
\begin{enumerate}
    \item Kommentar in das Textfeld eingeben
    \item \textbf{„Kommentar hinzufügen"} klicken
    \item Kommentar erscheint chronologisch
    \item Eigene Kommentare können gelöscht werden
\end{enumerate}

\subsection{Persönliche Artikelverwaltung}

\subsubsection{Meine Artikel}
\begin{enumerate}
    \item Navigation: \textbf{„Meine Artikel"} klicken
    \item Übersicht aller eigenen Artikel
    \item Bearbeiten und Löschen möglich
    \item Reservierungsstatus einsehen
\end{enumerate}

\subsubsection{Artikel bearbeiten}
\begin{enumerate}
    \item Artikel auswählen
    \item \textbf{„Bearbeiten"} klicken
    \item Formular mit aktuellen Daten öffnet sich
    \item Änderungen vornehmen
    \item \textbf{„Speichern"} klicken
\end{enumerate}

\subsubsection{Artikel löschen}
\begin{enumerate}
    \item Artikel auswählen
    \item \textbf{„Löschen"} klicken
    \item Bestätigung erforderlich
    \item Artikel wird permanent entfernt
\end{enumerate}

\begin{warningbox}
Gelöschte Artikel können nicht wiederhergestellt werden!
\end{warningbox}

\subsection{Profilverwaltung}

\subsubsection{Profil anzeigen}
\begin{enumerate}
    \item Navigation: \textbf{„Profil"} klicken
    \item Aktuelle Profildaten anzeigen
    \item Benutzerstatistiken einsehen
\end{enumerate}

\subsubsection{Profil bearbeiten}
\begin{enumerate}
    \item \textbf{„Profil bearbeiten"} klicken
    \item Daten ändern:
    \begin{itemize}
        \item Benutzername
        \item E-Mail-Adresse
        \item Passwort
        \item Profilbild
    \end{itemize}
    \item \textbf{„Speichern"} klicken
\end{enumerate}

\subsubsection{Avatar hochladen}
\begin{enumerate}
    \item Im Profil-Bearbeiten-Modus
    \item \textbf{„Avatar hochladen"} klicken
    \item Bilddatei auswählen (JPG, PNG)
    \item Automatische Größenanpassung
    \item Bild wird sofort angezeigt
\end{enumerate}

\section{Entwicklung}

\subsection{Entwicklungsumgebung}

\subsubsection{Hot Reload aktivieren}
\begin{lstlisting}[language=bash]
# Entwicklungsserver mit Hot Reload starten
./start-dev.sh

# Automatische Neuladen:
# - Frontend: Sofortige Updates bei Dateiänderungen
# - Backend: Automatischer Neustart bei Java-Änderungen
# - Datenbank: Schema-Updates via JPA
\end{lstlisting}

\subsubsection{Entwicklungstools}
\begin{itemize}
    \item \textbf{Frontend:} Vite Dev Server mit HMR
    \item \textbf{Backend:} Spring Boot DevTools
    \item \textbf{Datenbank:} PgAdmin Web-Interface
    \item \textbf{API-Testing:} Integrierte Swagger-UI
\end{itemize}

\subsection{Verfügbare Befehle}

\subsubsection{Hauptbefehle}
\begin{lstlisting}[language=bash]
# Alle Services starten
./start-dev.sh

# Mit PgAdmin Web-Interface
./start-dev.sh --with-pgadmin

# Clean Start (Container/Volumes löschen)
./start-dev.sh --clean

# Services stoppen
./start-dev.sh stop

# Alles neustarten
./start-dev.sh restart

# Logs anzeigen
./start-dev.sh logs

# Container-Status prüfen
./start-dev.sh status

# Zur Datenbank verbinden
./start-dev.sh db

# Komplette Bereinigung
./start-dev.sh clean
\end{lstlisting}

\subsubsection{Entwicklungsworkflow}
\begin{enumerate}
    \item Entwicklungsumgebung starten: \texttt{./start-dev.sh}
    \item Code editieren in \texttt{./Frontend/src} oder \texttt{./backend/src}
    \item Änderungen werden automatisch übernommen
    \item Testen im Browser: \url{http://localhost:5173}
    \item API testen: \url{http://localhost:8080}
\end{enumerate}

\subsection{Frontend-Entwicklung}

\subsubsection{Projektstruktur}
\begin{lstlisting}
Frontend/
├── src/
│   ├── components/        # Wiederverwendbare Komponenten
│   │   ├── auth/         # Authentifizierung
│   │   ├── HomePage/     # Homepage-spezifische Komponenten
│   │   └── ...
│   ├── pages/            # Seiten-Komponenten
│   ├── hooks/            # Custom React Hooks
│   ├── services/         # API-Services
│   ├── types/            # TypeScript-Definitionen
│   ├── contexts/         # React Contexts
│   └── App.tsx           # Haupt-App-Komponente
├── public/               # Statische Dateien
└── package.json          # Node.js Dependencies
\end{lstlisting}

\subsubsection{Neue Komponente hinzufügen}
\begin{enumerate}
    \item Komponente in \texttt{src/components/} erstellen
    \item TypeScript-Interface definieren
    \item Styling mit TailwindCSS
    \item In Parent-Komponente importieren
    \item Hot Reload zeigt Änderungen sofort
\end{enumerate}

\subsubsection{API-Integration}
\begin{lstlisting}[language=typescript]
// services/ItemService.ts
export const createItem = async (itemData: CreateItemRequest) => {
  const response = await axios.post('/api/items', itemData);
  return response.data;
};

// hooks/useAPI.ts
export const useCreateItem = () => {
  return useMutation({
    mutationFn: createItem,
    onSuccess: () => {
      queryClient.invalidateQueries(['items']);
    }
  });
};
\end{lstlisting}

\subsection{Backend-Entwicklung}

\subsubsection{Projektstruktur}
\begin{lstlisting}
backend/
├── src/main/java/com/lap/
│   ├── controller/       # REST-Controller
│   ├── service/          # Business Logic
│   ├── repository/       # Data Access
│   ├── entity/           # JPA-Entities
│   ├── dto/              # Data Transfer Objects
│   ├── config/           # Konfiguration
│   └── LapBackendApplication.java
├── src/main/resources/
│   ├── application.properties
│   └── db/init.sql
└── pom.xml               # Maven-Konfiguration
\end{lstlisting}

\subsubsection{Neue API-Endpoint hinzufügen}
\begin{enumerate}
    \item Controller-Methode erstellen
    \item Service-Logic implementieren
    \item Repository-Interface erweitern
    \item DTO für Request/Response definieren
    \item Validierung hinzufügen
    \item Tests schreiben
\end{enumerate}

\subsubsection{Beispiel: Neuer Endpoint}
\begin{lstlisting}[language=java]
@RestController
@RequestMapping("/api/items")
public class ItemController {

    @PostMapping
    public ResponseEntity<ItemResponse> createItem(
            @Valid @RequestBody CreateItemRequest request) {
        ItemResponse item = itemService.createItem(request);
        return ResponseEntity.ok(item);
    }
}
\end{lstlisting}

\section{Troubleshooting}

\subsection{Häufige Probleme}

\subsubsection{Container starten nicht}
\begin{lstlisting}[language=bash]
# Container-Status prüfen
podman ps -a

# Logs anzeigen
podman logs lap-backend
podman logs lap-frontend
podman logs lap-postgres

# Ports prüfen
ss -tulpn | grep :8080
ss -tulpn | grep :5173
\end{lstlisting}

\subsubsection{Datenbank-Verbindungsprobleme}
\begin{lstlisting}[language=bash]
# PostgreSQL-Status prüfen
podman exec lap-postgres pg_isready -U lapuser -d lapdb

# Datenbank-Verbindung testen
podman exec -it lap-postgres psql -U lapuser -d lapdb

# Netzwerk prüfen
podman network ls
podman network inspect lap-network
\end{lstlisting}

\subsubsection{Frontend lädt nicht}
\begin{lstlisting}[language=bash]
# Frontend-Container neu starten
podman restart lap-frontend

# Node.js-Abhängigkeiten neu installieren
cd Frontend
npm install
npm run dev

# Browser-Cache leeren
# Strg+F5 oder Entwicklertools -> Cache leeren
\end{lstlisting}

\subsubsection{Backend-API nicht erreichbar}
\begin{lstlisting}[language=bash]
# Backend-Container neu starten
podman restart lap-backend

# Java-Anwendung-Logs prüfen
podman logs lap-backend -f

# Port-Weiterleitung prüfen
podman port lap-backend

# API-Endpunkt testen
curl -X GET http://localhost:8080/api/items
\end{lstlisting}

\subsection{Fehlerdiagnose}

\subsubsection{Systematische Problemlösung}
\begin{enumerate}
    \item \textbf{Container-Status:} Alle Container laufen?
    \item \textbf{Logs prüfen:} Fehlermeldungen in den Logs?
    \item \textbf{Netzwerk:} Container können kommunizieren?
    \item \textbf{Ports:} Sind alle Ports frei und weitergeleitet?
    \item \textbf{Volumes:} Sind Dateien korrekt gemountet?
\end{enumerate}

\subsubsection{Clean Restart}
\begin{lstlisting}[language=bash]
# Alles stoppen und bereinigen
./start-dev.sh stop
./start-dev.sh clean

# Komplett neu starten
./start-dev.sh
\end{lstlisting}

\subsubsection{Einzelne Services debuggen}
\begin{lstlisting}[language=bash]
# Nur Backend starten
cd backend
podman build -t lap-backend .
podman run -it lap-backend

# Nur Frontend starten
cd Frontend
npm run dev

# Nur Datenbank starten
podman run -d --name lap-postgres \
  -e POSTGRES_DB=lapdb \
  -e POSTGRES_USER=lapuser \
  -e POSTGRES_PASSWORD=lappassword \
  -p 5432:5432 \
  postgres:14
\end{lstlisting}

\section{API-Dokumentation}

\subsection{Authentifizierung}

\subsubsection{Benutzerregistrierung}
\begin{lstlisting}[language=bash]
curl -X POST http://localhost:8080/api/auth/register \
  -H "Content-Type: application/json" \
  -d '{
    "username": "newuser",
    "email": "user@example.com",
    "password": "password123"
  }'
\end{lstlisting}

\subsubsection{Benutzeranmeldung}
\begin{lstlisting}[language=bash]
curl -X POST http://localhost:8080/api/auth/login \
  -H "Content-Type: application/json" \
  -d '{
    "username": "testuser",
    "password": "password123"
  }'
\end{lstlisting}

\subsubsection{Geschützte Endpunkte}
\begin{lstlisting}[language=bash]
# JWT-Token verwenden
curl -X GET http://localhost:8080/api/auth/me \
  -H "Authorization: Bearer YOUR_JWT_TOKEN"
\end{lstlisting}

\subsection{Artikel-API}

\subsubsection{Artikel auflisten}
\begin{lstlisting}[language=bash]
# Alle Artikel
curl -X GET http://localhost:8080/api/items

# Mit Paginierung
curl -X GET "http://localhost:8080/api/items?page=0&size=10"

# Mit Suche
curl -X GET "http://localhost:8080/api/items/search?query=laptop"
\end{lstlisting}

\subsubsection{Artikel erstellen}
\begin{lstlisting}[language=bash]
curl -X POST http://localhost:8080/api/items \
  -H "Content-Type: application/json" \
  -H "Authorization: Bearer YOUR_JWT_TOKEN" \
  -d '{
    "title": "Alter Laptop",
    "description": "Funktioniert noch gut",
    "category": "Elektronik",
    "location": "Berlin",
    "condition": "Gut"
  }'
\end{lstlisting}

\subsubsection{Artikel mit Bildern}
\begin{lstlisting}[language=bash]
curl -X POST http://localhost:8080/api/items \
  -H "Authorization: Bearer YOUR_JWT_TOKEN" \
  -F "title=Laptop" \
  -F "description=Beschreibung" \
  -F "category=Elektronik" \
  -F "location=Berlin" \
  -F "condition=Gut" \
  -F "images=@/path/to/image1.jpg" \
  -F "images=@/path/to/image2.jpg"
\end{lstlisting}

\subsection{Datei-API}

\subsubsection{Datei hochladen}
\begin{lstlisting}[language=bash]
curl -X POST http://localhost:8080/api/files/upload \
  -H "Authorization: Bearer YOUR_JWT_TOKEN" \
  -F "file=@/path/to/your/file.jpg"
\end{lstlisting}

\subsubsection{Datei herunterladen}
\begin{lstlisting}[language=bash]
curl -X GET http://localhost:8080/api/files/download/filename.jpg \
  -o downloaded_file.jpg
\end{lstlisting}

\subsubsection{Download-URL generieren}
\begin{lstlisting}[language=bash]
curl -X GET http://localhost:8080/api/files/url/filename.jpg
\end{lstlisting}

\section{Sicherheit}

\subsection{Authentifizierung}

\subsubsection{JWT-Token}
\begin{itemize}
    \item Tokens haben eine Ablaufzeit von 24 Stunden
    \item Automatische Erneuerung bei aktiver Nutzung
    \item Sichere Speicherung im Browser
    \item Automatische Abmeldung bei Ablauf
\end{itemize}

\subsubsection{Passwort-Sicherheit}
\begin{itemize}
    \item Mindestens 6 Zeichen erforderlich
    \item BCrypt-Verschlüsselung mit Strength 12
    \item Keine Passwort-Speicherung im Klartext
    \item Sichere Passwort-Reset-Funktion
\end{itemize}

\subsection{Datenschutz}

\subsubsection{Persönliche Daten}
\begin{itemize}
    \item Minimale Datensammlung
    \item Verschlüsselte Speicherung
    \item Keine Weitergabe an Dritte
    \item Benutzer haben Kontrolle über ihre Daten
\end{itemize}

\subsubsection{Datenlöschung}
\begin{itemize}
    \item Artikel können jederzeit gelöscht werden
    \item Bilder werden automatisch entfernt
    \item Account-Löschung auf Anfrage
    \item Sichere Datenlöschung aus Cloud-Speicher
\end{itemize}

\section{Performance und Optimierung}

\subsection{Ladezeiten}
\begin{itemize}
    \item Optimierte Bilder mit automatischer Komprimierung
    \item Lazy Loading für große Bildergalerien
    \item Effiziente Datenbankabfragen
    \item Frontend-Caching mit React Query
\end{itemize}

\subsection{Responsive Design}
\begin{itemize}
    \item Optimiert für alle Bildschirmgrößen
    \item Touch-freundliche Bedienung auf Mobilgeräten
    \item Adaptive Layouts für verschiedene Geräte
    \item Schnelle Navigation auf allen Plattformen
\end{itemize}

\section{Backup und Wiederherstellung}

\subsection{Datensicherung}
\begin{itemize}
    \item Automatische tägliche Datenbank-Backups
    \item Cloud-Speicher mit integrierter Redundanz
    \item Versionskontrolle für Code-Änderungen
    \item Konfigurationsdaten in Git
\end{itemize}

\subsection{Wiederherstellung}
\begin{itemize}
    \item Schnelle Wiederherstellung aus Backups
    \item Point-in-Time-Recovery für Datenbank
    \item Automatische Failover-Mechanismen
    \item Dokumentierte Recovery-Procedures
\end{itemize}

\section{Erweiterte Funktionen}

\subsection{Zukünftige Features}
\begin{itemize}
    \item Push-Benachrichtigungen für neue Artikel
    \item Erweiterte Suchfilter (Preis, Entfernung)
    \item Bewertungssystem für Benutzer
    \item Chat-Funktion zwischen Benutzern
    \item Mobile App für iOS und Android
    \item Integration mit sozialen Medien
\end{itemize}

\subsection{API-Erweiterungen}
\begin{itemize}
    \item GraphQL-API für flexible Datenabfragen
    \item WebSocket-Support für Real-time Updates
    \item OAuth-Integration für Social Login
    \item Webhook-Support für externe Integrationen
\end{itemize}

\section{Monitoring und Logs}

\subsection{Anwendungsüberwachung}
\begin{itemize}
    \item Health-Check-Endpunkte für alle Services
    \item Performance-Metriken und Alerts
    \item Fehlerberichterstattung mit Stack Traces
    \item Benutzeraktivitäts-Tracking
\end{itemize}

\subsection{Log-Management}
\begin{itemize}
    \item Strukturierte Logs in JSON-Format
    \item Zentrale Log-Aggregation
    \item Automatische Log-Rotation
    \item Suchbare Log-Historie
\end{itemize}

\section{Deployment}

\subsection{Produktionsumgebung}
\begin{itemize}
    \item Multi-Container-Deployment mit Docker Compose
    \item Reverse Proxy mit Nginx
    \item SSL/TLS-Terminierung
    \item Load Balancing für Hochverfügbarkeit
\end{itemize}

\subsection{CI/CD Pipeline}
\begin{itemize}
    \item Automatische Builds bei Code-Änderungen
    \item Umfassende Test-Suites
    \item Automatische Sicherheitstests
    \item Zero-Downtime-Deployments
\end{itemize}

\section{Support und Wartung}

\subsection{Bekannte Einschränkungen}
\begin{itemize}
    \item Maximale Dateigröße: 10 MB pro Bild
    \item Maximale Anzahl Bilder: 5 pro Artikel
    \item Unterstützte Bildformate: JPG, PNG, GIF
    \item Keine Offline-Funktionalität
\end{itemize}

\subsection{Geplante Verbesserungen}
\begin{itemize}
    \item Erhöhung der Dateigrößenlimits
    \item Unterstützung für Video-Uploads
    \item Progressive Web App (PWA) Features
    \item Verbessertes Caching für bessere Performance
\end{itemize}

\section{Anhang}

\subsection{Kommandoreferenz}
\begin{longtable}{|p{0.4\textwidth}|p{0.5\textwidth}|}
\hline
\textbf{Kommando} & \textbf{Beschreibung} \\
\hline
\texttt{./start-dev.sh} & Alle Services starten \\
\hline
\texttt{./start-dev.sh --with-pgadmin} & Mit PgAdmin Web-Interface \\
\hline
\texttt{./start-dev.sh --clean} & Clean Start mit neuen Containern \\
\hline
\texttt{./start-dev.sh stop} & Alle Services stoppen \\
\hline
\texttt{./start-dev.sh restart} & Services neustarten \\
\hline
\texttt{./start-dev.sh logs} & Logs anzeigen \\
\hline
\texttt{./start-dev.sh status} & Container-Status prüfen \\
\hline
\texttt{./start-dev.sh db} & Zur Datenbank verbinden \\
\hline
\texttt{./start-dev.sh clean} & Komplette Bereinigung \\
\hline
\end{longtable}

\subsection{Umgebungsvariablen}
\begin{longtable}{|p{0.3\textwidth}|p{0.25\textwidth}|p{0.35\textwidth}|}
\hline
\textbf{Variable} & \textbf{Standard} & \textbf{Beschreibung} \\
\hline
\texttt{VITE\_API\_URL} & http://localhost:8080 & Backend-API URL \\
\hline
\texttt{JWT\_SECRET} & mySecretKey... & JWT-Signatur-Schlüssel \\
\hline
\texttt{JWT\_EXPIRATION} & 86400000 & Token-Ablaufzeit in ms \\
\hline
\texttt{B2\_APPLICATION\_KEY\_ID} & - & Backblaze B2 Key ID \\
\hline
\texttt{B2\_APPLICATION\_KEY} & - & Backblaze B2 Key \\
\hline
\texttt{B2\_BUCKET\_NAME} & - & Backblaze B2 Bucket \\
\hline
\end{longtable}

\subsection{Nützliche Links}
\begin{itemize}
    \item \textbf{React Documentation:} \url{https://react.dev/}
    \item \textbf{Spring Boot Guides:} \url{https://spring.io/guides}
    \item \textbf{PostgreSQL Documentation:} \url{https://www.postgresql.org/docs/}
    \item \textbf{TailwindCSS Documentation:} \url{https://tailwindcss.com/docs}
    \item \textbf{DaisyUI Components:} \url{https://daisyui.com/components/}
    \item \textbf{Backblaze B2 Documentation:} \url{https://www.backblaze.com/b2/docs/}
    \item \textbf{Podman Documentation:} \url{https://podman.io/docs}
\end{itemize}

\subsection{Troubleshooting-Checkliste}
\begin{enumerate}
    \item Sind alle Container gestartet? (\texttt{podman ps})
    \item Sind die Ports frei? (\texttt{ss -tulpn | grep :8080})
    \item Funktioniert die Datenbankverbindung? (\texttt{./start-dev.sh db})
    \item Sind die Logs fehlerfrei? (\texttt{./start-dev.sh logs})
    \item Ist die Netzwerkkonnektivität OK? (\texttt{curl localhost:8080})
    \item Sind die Volumes korrekt gemountet? (\texttt{podman inspect})
    \item Sind alle Abhängigkeiten installiert? (Container neu bauen)
    \item Hilft ein Clean Restart? (\texttt{./start-dev.sh clean})
\end{enumerate}

\section{Schlusswort}

Die LAP-Verschenke-Plattform demonstriert eine moderne, vollständige Full-Stack-Entwicklung mit aktuellen Technologien und Best Practices. Die Anwendung zeigt:

\begin{itemize}
    \item Professionelle Softwareentwicklung mit TypeScript und Java
    \item Sichere Authentifizierung und Datenschutz
    \item Benutzerfreundliche, responsive Web-Oberfläche
    \item Skalierbare Cloud-Integration
    \item Containerisierte Entwicklungsumgebung
    \item Umfassende Dokumentation und Testing
\end{itemize}

Diese Dokumentation dient als Referenz für Entwickler, Benutzer und Prüfer der Lehrabschlussprüfung. Das Projekt zeigt die Fähigkeit zur selbstständigen Entwicklung komplexer Webanwendungen und die Anwendung moderner Entwicklungsmethoden.

\textbf{Viel Erfolg bei der Nutzung und Weiterentwicklung der LAP-Plattform!}

\end{document}
