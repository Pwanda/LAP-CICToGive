\documentclass[a4paper,12pt]{article}
\usepackage[utf8]{inputenc}
\usepackage[ngerman]{babel}
\usepackage{geometry}
\usepackage{fancyhdr}
\usepackage{graphicx}
\usepackage{booktabs}
\usepackage{enumitem}
\usepackage{xcolor}
\usepackage{hyperref}
\usepackage{longtable}
\usepackage{array}
\usepackage{tabularx}
\usepackage{ltxtable}

\geometry{a4paper, left=3cm, right=2cm, top=2.5cm, bottom=2.5cm}

\pagestyle{fancy}
\fancyhf{}
\fancyhead[L]{LAP - Verschenke-Plattform}
\fancyhead[R]{Lastenheft}
\fancyfoot[C]{\thepage}

\hypersetup{
    colorlinks=true,
    linkcolor=blue,
    filecolor=magenta,
    urlcolor=cyan,
    pdftitle={Lastenheft - LAP Verschenke-Plattform},
    pdfauthor={Prüfungskandidat},
    pdfsubject={Lehrabschlussprüfung Applikationsentwicklung},
    pdfkeywords={Lastenheft, Full-Stack, React, Spring Boot, PostgreSQL}
}

\title{
    \huge\textbf{Lastenheft}\\
    \Large LAP - Verschenke-Plattform\\
    \large Eine moderne Full-Stack-Webanwendung
}
\author{Lehrabschlussprüfung Applikationsentwicklung}
\date{\today}

\begin{document}

\maketitle

\newpage

\tableofcontents

\newpage

\section{Projektübersicht}

\subsection{Projektbezeichnung}
\textbf{LAP - Local Application Platform}\\
Eine webbasierte Plattform zum Verschenken und Tauschen von Gegenständen

\subsection{Projektziel}
Die Entwicklung einer modernen, benutzerfreundlichen Webanwendung, die es Nutzern ermöglicht, nicht mehr benötigte Gegenstände zu verschenken anstatt sie wegzuwerfen. Die Plattform soll nachhaltiges Verhalten fördern und eine lokale Gemeinschaft zum Teilen von Ressourcen schaffen.

\subsection{Zielgruppe}
\begin{itemize}
    \item Privatpersonen, die Gegenstände verschenken möchten
    \item Personen, die nach kostenlosen Artikeln suchen
    \item Umweltbewusste Nutzer, die Nachhaltigkeit fördern
    \item Lokale Gemeinschaften und Nachbarschaften
\end{itemize}

\subsection{Projekthintergrund}
In der heutigen Konsumgesellschaft werden viele funktionsfähige Gegenstände weggeworfen, obwohl sie für andere noch nützlich wären. Diese Plattform schafft eine digitale Lösung für das Problem der Ressourcenverschwendung und fördert eine Kreislaufwirtschaft auf lokaler Ebene.

\section{Funktionale Anforderungen}

\subsection{Benutzerverwaltung}
\begin{longtable}{|p{0.15\textwidth}|p{0.35\textwidth}|p{0.4\textwidth}|}
\hline
\textbf{ID} & \textbf{Anforderung} & \textbf{Beschreibung} \\
\hline
FA-01 & Benutzerregistrierung & Neue Nutzer können sich mit E-Mail-Adresse, Benutzername und Passwort registrieren \\
\hline
FA-02 & Benutzeranmeldung & Registrierte Nutzer können sich mit Benutzername/E-Mail und Passwort anmelden \\
\hline
FA-03 & JWT-Authentifizierung & Sichere Authentifizierung mittels JSON Web Tokens \\
\hline
FA-04 & Profilmanagement & Nutzer können ihre Profildaten einsehen und bearbeiten \\
\hline
FA-05 & Avatar-Upload & Nutzer können ein Profilbild hochladen \\
\hline
FA-06 & Passwort-Sicherheit & Passwörter werden verschlüsselt gespeichert (BCrypt) \\
\hline
FA-07 & Benutzerabmeldung & Nutzer können sich sicher abmelden \\
\hline
\end{longtable}

\subsection{Artikelverwaltung}
\begin{longtable}{|p{0.15\textwidth}|p{0.35\textwidth}|p{0.4\textwidth}|}
\hline
\textbf{ID} & \textbf{Anforderung} & \textbf{Beschreibung} \\
\hline
FA-08 & Artikel erstellen & Nutzer können neue Artikel mit Titel, Beschreibung, Kategorie, Standort und Zustand erstellen \\
\hline
FA-09 & Bildupload & Mehrere Bilder pro Artikel können hochgeladen werden \\
\hline
FA-10 & Artikel anzeigen & Alle verfügbaren Artikel werden in einer übersichtlichen Galerie dargestellt \\
\hline
FA-11 & Artikeldetails & Detailansicht eines Artikels mit allen Informationen und Bildern \\
\hline
FA-12 & Artikel bearbeiten & Artikelbesitzer können ihre Artikel bearbeiten \\
\hline
FA-13 & Artikel löschen & Artikelbesitzer können ihre Artikel löschen \\
\hline
FA-14 & Artikel reservieren & Nutzer können Interesse an einem Artikel bekunden \\
\hline
FA-15 & Meine Artikel & Nutzer können ihre eigenen Artikel verwalten \\
\hline
\end{longtable}

\subsection{Such- und Filterfunktionen}
\begin{longtable}{|p{0.15\textwidth}|p{0.35\textwidth}|p{0.4\textwidth}|}
\hline
\textbf{ID} & \textbf{Anforderung} & \textbf{Beschreibung} \\
\hline
FA-16 & Textsuche & Nutzer können nach Artikeln anhand des Titels oder der Beschreibung suchen \\
\hline
FA-17 & Kategoriefilter & Filterung nach Kategorien (Elektronik, Möbel, Kleidung, etc.) \\
\hline
FA-18 & Standortfilter & Suche nach Artikeln in bestimmten Standorten \\
\hline
FA-19 & Zustandsfilter & Filterung nach Zustand (Neu, Gut, Gebraucht) \\
\hline
FA-20 & Kombinierte Filter & Mehrere Filter können gleichzeitig angewendet werden \\
\hline
FA-21 & Sortierung & Artikel können nach Datum, Relevanz sortiert werden \\
\hline
\end{longtable}

\subsection{Kommentarsystem}
\begin{longtable}{|p{0.15\textwidth}|p{0.35\textwidth}|p{0.4\textwidth}|}
\hline
\textbf{ID} & \textbf{Anforderung} & \textbf{Beschreibung} \\
\hline
FA-22 & Kommentare erstellen & Nutzer können Kommentare zu Artikeln verfassen \\
\hline
FA-23 & Kommentare anzeigen & Alle Kommentare zu einem Artikel werden chronologisch angezeigt \\
\hline
FA-24 & Kommentare löschen & Nutzer können ihre eigenen Kommentare löschen \\
\hline
FA-25 & Kommentar-Validierung & Kommentare werden auf Länge und Inhalt validiert \\
\hline
\end{longtable}

\subsection{Dateiverwaltung}
\begin{longtable}{|p{0.15\textwidth}|p{0.35\textwidth}|p{0.4\textwidth}|}
\hline
\textbf{ID} & \textbf{Anforderung} & \textbf{Beschreibung} \\
\hline
FA-26 & Cloud-Speicher & Integration mit Backblaze B2 für Bildspeicherung \\
\hline
FA-27 & Mehrfach-Upload & Mehrere Dateien können gleichzeitig hochgeladen werden \\
\hline
FA-28 & Bildoptimierung & Automatische Bildgrößenanpassung und Komprimierung \\
\hline
FA-29 & Datei-Validierung & Überprüfung von Dateityp und -größe \\
\hline
FA-30 & Sichere URLs & Generierung sicherer Download-URLs \\
\hline
\end{longtable}

\section{Nicht-funktionale Anforderungen}

\subsection{Leistungsanforderungen}
\begin{longtable}{|p{0.15\textwidth}|p{0.35\textwidth}|p{0.4\textwidth}|}
\hline
\textbf{ID} & \textbf{Anforderung} & \textbf{Beschreibung} \\
\hline
NFA-01 & Antwortzeit & Seitenaufbau in unter 2 Sekunden \\
\hline
NFA-02 & Gleichzeitige Nutzer & System unterstützt mindestens 100 gleichzeitige Benutzer \\
\hline
NFA-03 & Datenbankperformance & Datenbankabfragen unter 500ms \\
\hline
NFA-04 & Bildladezeit & Bilder laden in unter 3 Sekunden \\
\hline
NFA-05 & Hot Reload & Entwicklungsumgebung mit Hot Reload für effiziente Entwicklung \\
\hline
\end{longtable}

\subsection{Sicherheitsanforderungen}
\begin{longtable}{|p{0.15\textwidth}|p{0.35\textwidth}|p{0.4\textwidth}|}
\hline
\textbf{ID} & \textbf{Anforderung} & \textbf{Beschreibung} \\
\hline
NFA-06 & Passwort-Verschlüsselung & BCrypt-Verschlüsselung für Passwörter \\
\hline
NFA-07 & JWT-Sicherheit & Sichere JWT-Token mit Ablaufzeit \\
\hline
NFA-08 & CORS-Konfiguration & Korrekte Cross-Origin-Konfiguration \\
\hline
NFA-09 & Input-Validierung & Umfassende Server- und Client-seitige Validierung \\
\hline
NFA-10 & SQL-Injection-Schutz & Verwendung von JPA/Hibernate für sichere Datenbankzugriffe \\
\hline
NFA-11 & Authentifizierung & Geschützte Routen erfordern gültige Authentifizierung \\
\hline
\end{longtable}

\subsection{Usability-Anforderungen}
\begin{longtable}{|p{0.15\textwidth}|p{0.35\textwidth}|p{0.4\textwidth}|}
\hline
\textbf{ID} & \textbf{Anforderung} & \textbf{Beschreibung} \\
\hline
NFA-12 & Responsive Design & Optimierung für Desktop, Tablet und Mobile \\
\hline
NFA-13 & Intuitive Navigation & Benutzerfreundliche Navigation mit klarer Struktur \\
\hline
NFA-14 & Barrierefreiheit & Grundlegende Accessibility-Standards \\
\hline
NFA-15 & Mehrsprachigkeit & Grundstruktur für Deutsche Lokalisierung \\
\hline
NFA-16 & Modernes Design & Attraktives UI mit DaisyUI und TailwindCSS \\
\hline
\end{longtable}

\subsection{Technische Anforderungen}
\begin{longtable}{|p{0.15\textwidth}|p{0.35\textwidth}|p{0.4\textwidth}|}
\hline
\textbf{ID} & \textbf{Anforderung} & \textbf{Beschreibung} \\
\hline
NFA-17 & Browser-Kompatibilität & Unterstützung moderner Browser (Chrome, Firefox, Safari, Edge) \\
\hline
NFA-18 & Containerisierung & Vollständige Containerisierung mit Podman/Docker \\
\hline
NFA-19 & Datenpersistierung & Zuverlässige Datenspeicherung in PostgreSQL \\
\hline
NFA-20 & API-Dokumentation & RESTful API mit klarer Dokumentation \\
\hline
NFA-21 & Fehlerbehandlung & Umfassende Fehlerbehandlung und Logging \\
\hline
\end{longtable}

\section{Systemarchitektur}

\subsection{Architektur-Übersicht}
Das System folgt einer modernen Full-Stack-Architektur mit klarer Trennung zwischen Frontend, Backend und Datenbank:

\begin{itemize}
    \item \textbf{Frontend:} React 19 mit TypeScript für eine typsichere, moderne Benutzeroberfläche
    \item \textbf{Backend:} Spring Boot 3.2 mit Java 17 für robuste Server-Logik
    \item \textbf{Datenbank:} PostgreSQL für zuverlässige Datenpersistierung
    \item \textbf{Cloud-Speicher:} Backblaze B2 für skalierbare Dateispeicherung
    \item \textbf{Containerisierung:} Podman für konsistente Entwicklungs- und Deployment-Umgebungen
\end{itemize}

\subsection{Technologie-Stack}

\subsubsection{Frontend-Technologien}
\begin{itemize}
    \item React 19 - Moderne JavaScript-Bibliothek für Benutzeroberflächen
    \item TypeScript - Typsicherheit und bessere Entwicklererfahrung
    \item Vite - Schneller Build-Tool und Entwicklungsserver
    \item React Router v7 - Client-seitiges Routing
    \item React Query - Datenmanagement und Caching
    \item TailwindCSS - Utility-first CSS Framework
    \item DaisyUI - Komponenten-Bibliothek für TailwindCSS
    \item React Hook Form - Effiziente Formularvalidierung
    \item Zod - Schema-Validierung
\end{itemize}

\subsubsection{Backend-Technologien}
\begin{itemize}
    \item Spring Boot 3.2 - Enterprise-Java-Framework
    \item Spring Security - Authentifizierung und Autorisierung
    \item Spring Data JPA - Datenzugriff und ORM
    \item JWT (JSON Web Tokens) - Sichere Token-basierte Authentifizierung
    \item PostgreSQL - Relationale Datenbank
    \item Hibernate - ORM-Framework
    \item Maven - Build-Management
    \item Backblaze B2 SDK - Cloud-Speicher Integration
\end{itemize}

\subsubsection{Entwicklungstools}
\begin{itemize}
    \item Podman - Containerisierung
    \item Git - Versionskontrolle
    \item ESLint - Code-Qualität für JavaScript/TypeScript
    \item Prettier - Code-Formatierung
    \item Spring Boot DevTools - Hot Reload für Backend
\end{itemize}

\section{Benutzerschnittstellen}

\subsection{Hauptseiten}
\begin{longtable}{|p{0.2\textwidth}|p{0.35\textwidth}|p{0.35\textwidth}|}
\hline
\textbf{Seite} & \textbf{Beschreibung} & \textbf{Funktionen} \\
\hline
Landing Page & Startseite für nicht angemeldete Benutzer & Überblick, Anmeldung, Registrierung \\
\hline
Home Page & Hauptseite mit Artikelübersicht & Artikel anzeigen, Suchen, Filtern, Paginierung \\
\hline
Artikel-Detail & Detailansicht eines Artikels & Vollständige Informationen, Bilder, Kommentare \\
\hline
Meine Artikel & Verwaltung eigener Artikel & Eigene Artikel anzeigen, bearbeiten, löschen \\
\hline
Profil & Benutzerprofil-Verwaltung & Profildaten bearbeiten, Avatar hochladen \\
\hline
\end{longtable}

\subsection{Modale Dialoge}
\begin{longtable}{|p{0.2\textwidth}|p{0.35\textwidth}|p{0.35\textwidth}|}
\hline
\textbf{Modal} & \textbf{Beschreibung} & \textbf{Funktionen} \\
\hline
Artikel erstellen & Neuen Artikel hinzufügen & Formulareingabe, Bildupload, Validierung \\
\hline
Artikel bearbeiten & Bestehenden Artikel ändern & Daten aktualisieren, Bilder verwalten \\
\hline
Artikel-Detail & Artikel in Großansicht & Bildergalerie, Kommentare, Reservierung \\
\hline
\end{longtable}

\subsection{Navigation}
\begin{itemize}
    \item \textbf{Hauptnavigation:} Logo, Menüpunkte, Benutzerstatus
    \item \textbf{Suchleiste:} Prominente Platzierung für Artikelsuche
    \item \textbf{Kategoriefilter:} Horizontale Filterleiste
    \item \textbf{Floating Action Button:} Schneller Zugriff zum Artikel erstellen
    \item \textbf{Footer:} Zusätzliche Links und Informationen
\end{itemize}

\section{Datenmodell}

\subsection{Hauptentitäten}
\begin{longtable}{|p{0.2\textwidth}|p{0.35\textwidth}|p{0.35\textwidth}|}
\hline
\textbf{Entität} & \textbf{Beschreibung} & \textbf{Hauptattribute} \\
\hline
User & Benutzer des Systems & id, username, email, password, avatarUrl, createdAt, updatedAt \\
\hline
Item & Artikel/Gegenstände & id, title, description, category, location, condition, imageUrls, isReserved, userId \\
\hline
Comment & Kommentare zu Artikeln & id, content, itemId, userId, createdAt \\
\hline
\end{longtable}

\subsection{Beziehungen}
\begin{itemize}
    \item \textbf{User : Item} - 1:n (Ein Benutzer kann mehrere Artikel haben)
    \item \textbf{Item : Comment} - 1:n (Ein Artikel kann mehrere Kommentare haben)
    \item \textbf{User : Comment} - 1:n (Ein Benutzer kann mehrere Kommentare schreiben)
\end{itemize}

\section{Qualitätsanforderungen}

\subsection{Testanforderungen}
\begin{itemize}
    \item Unit-Tests für kritische Backend-Funktionen
    \item Integration-Tests für API-Endpunkte
    \item Frontend-Tests für Benutzerinteraktionen
    \item Sicherheitstests für Authentifizierung
    \item Performance-Tests für Datenbankabfragen
\end{itemize}

\subsection{Dokumentationsanforderungen}
\begin{itemize}
    \item Umfassende API-Dokumentation
    \item Benutzerhandbuch
    \item Entwickler-Dokumentation
    \item Deployment-Anleitung
    \item Troubleshooting-Guide
\end{itemize}

\section{Projektabgrenzung}

\subsection{Im Projektumfang enthalten}
\begin{itemize}
    \item Vollständige Webanwendung mit Frontend und Backend
    \item Benutzerverwaltung mit Authentifizierung
    \item Artikelverwaltung mit Bildupload
    \item Such- und Filterfunktionen
    \item Kommentarsystem
    \item Responsive Design
    \item Containerisierte Entwicklungsumgebung
    \item Cloud-Speicher Integration
\end{itemize}

\subsection{Nicht im Projektumfang enthalten}
\begin{itemize}
    \item Mobile Apps (nur responsive Web-App)
    \item Zahlungssystem (nur Verschenken, kein Verkauf)
    \item Komplexes Benachrichtigungssystem
    \item Social Media Integration
    \item Erweiterte Analytics
    \item Multi-Mandantenfähigkeit
    \item Offline-Funktionalität
\end{itemize}

\section{Risiken und Annahmen}

\subsection{Projektrisiken}
\begin{longtable}{|p{0.15\textwidth}|p{0.35\textwidth}|p{0.4\textwidth}|}
\hline
\textbf{Risiko} & \textbf{Beschreibung} & \textbf{Maßnahme} \\
\hline
Technische Komplexität & Integration verschiedener Technologien & Proof of Concept, schrittweise Entwicklung \\
\hline
Sicherheitslücken & Unsichere Authentifizierung oder Datenlecks & Security-Reviews, bewährte Praktiken \\
\hline
Performance-Probleme & Langsame Ladezeiten bei vielen Bildern & Optimierung, Caching, CDN \\
\hline
Browser-Kompatibilität & Funktioniert nicht in älteren Browsern & Testing, Polyfills falls nötig \\
\hline
\end{longtable}

\subsection{Annahmen}
\begin{itemize}
    \item Nutzer haben Zugang zu modernen Webbrowsern
    \item Internetverbindung ist verfügbar
    \item Backblaze B2 Service ist verfügbar und zuverlässig
    \item PostgreSQL-Datenbank ist verfügbar
    \item Containerisierung ist in der Deployment-Umgebung möglich
\end{itemize}

\section{Erfolgskriterien}

\subsection{Technische Erfolgskriterien}
\begin{itemize}
    \item Vollständig funktionsfähige Webanwendung
    \item Alle funktionalen Anforderungen implementiert
    \item Responsive Design funktioniert auf allen Geräten
    \item Sicherheitsanforderungen erfüllt
    \item Performance-Ziele erreicht
    \item Automatisierte Deployment-Pipeline
\end{itemize}

\subsection{Qualitätskriterien}
\begin{itemize}
    \item Code-Qualität entspricht Best Practices
    \item Umfassende Dokumentation vorhanden
    \item Testabdeckung für kritische Funktionen
    \item Benutzerfreundliche Oberfläche
    \item Stabile und wartbare Codebasis
\end{itemize}

\section{Anhang}

\subsection{Glossar}
\begin{longtable}{|p{0.2\textwidth}|p{0.7\textwidth}|}
\hline
\textbf{Begriff} & \textbf{Definition} \\
\hline
API & Application Programming Interface - Schnittstelle für Anwendungen \\
\hline
JWT & JSON Web Token - Standard für sichere Token-basierte Authentifizierung \\
\hline
ORM & Object-Relational Mapping - Abbildung von Objekten auf Datenbankstrukturen \\
\hline
SPA & Single Page Application - Webanwendung mit dynamischem Inhalt \\
\hline
CORS & Cross-Origin Resource Sharing - Sicherheitsmechanismus für Web-APIs \\
\hline
REST & Representational State Transfer - Architekturstil für Web-Services \\
\hline
\end{longtable}

\subsection{Referenzen}
\begin{itemize}
    \item React Documentation: \url{https://reactjs.org/docs/}
    \item Spring Boot Documentation: \url{https://spring.io/projects/spring-boot}
    \item PostgreSQL Documentation: \url{https://www.postgresql.org/docs/}
    \item Backblaze B2 API: \url{https://www.backblaze.com/b2/docs/}
    \item TailwindCSS Documentation: \url{https://tailwindcss.com/docs}
\end{itemize}

\end{document}
